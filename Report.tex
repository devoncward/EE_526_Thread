\documentclass[letterpaper,twocolumn,10pt]{article}
\usepackage{epsfig,xspace,url}
\usepackage{authblk}

\usepackage{graphicx}
\graphicspath{ {./images/} }

\usepackage[
backend=biber,
style=alphabetic,
sorting=ynt
]{biblatex}

\addbibresource{references.bib}

\title{Thread Network In Clyde Building Report}
\author{Devon Ward, Daniel Harman}

\begin{document}

\maketitle

\subsection*{Introduction}

Thread is a mesh networking protocol which has been developed mainly for smart home devices \cite*{thread_website}. The main idea is to remove a single point of failure for the networking smart home devices, allowing them to communicate with each other and avoid having a potential problem where one device fails, taking down the entire network. Another beneficial aspect of thread is that it is designed to be low power \cite*{Thread_paper_1}, which is also a really good thing for IoT devices, which are often power constrained. 

One problem that Thread networks have currently is that they are not very well tested. There are many sources that have successfully tested Thread networks and proven that they can operate in a noisy environment \cite*{Thread_noisy_env} but very little to classify their performance in indoor urban environments. Work has previously been done to quantify the throughput of Thread networks in an indoor area \cite*{Thread_on_drones} but most publications are focused on throughput achievements, rather than distance and reliability.  

Our project seeks to attempt to classify the performance of Thread in the Clyde engineering building at Brigham Young University, and determine if a Thread network is capable of allowing two users to communicate from the graduate research lab in CB 480 to anywhere in the Clyde building. In doing so, we also seek to understand how many nodes are required to allow for uninterrupted communication in the Clyde building. 

Our experiments seek to quantify the range of the Thread devices using ESPRESSIF ESP32-H2 Dev boards \cite*{ESP32_h_board} to get an estimate of how many routers we will require for the Thread network. We then plan to strategically place the routers throughout the Clyde building, and attempt to walk through the building without losing connection to the parent node in the graduate lab. We will quantify exactly how many router Thread nodes we will require, and attempt to use the minimum number possible.

We found that we only required one Thread router node, one parent node, and one child node to allow the child node to go all around the basement of the Clyde, and through the rest of the building without losing connection. We were also able to briefly go outside the Clyde building and still communicate with the graduate lab. 

\subsection*{Related Work}

Summarize the existing work related to your problem.

\subsection*{Methodology}

Describe your solution and any other methods you have used to solve your problem.

\subsection*{Implementation/Experimentation}

Describe your experiments and your results. Discuss your results.

\subsection*{Conclusion}

Summarize your work and your results. Indicated any future directions.

\subsection*{References}

\printbibliography

\end{document}